\documentclass[]{article}
\usepackage{graphicx}
\usepackage{fullpage}
\usepackage{blindtext}
\usepackage{amsmath}
\title{Quick and dirty exponential function}
\author{Anders Bak-Nyhus}
\date{}
\begin{document}
\maketitle

\section*{The exponential function}
An exponential function, is a function of the form
\begin{equation}
	f(x) = ab^x
\end{equation}
A more interesting case of exponential functions, is the one with the base e, where the exponential function is it's own derivative
\begin{equation}
	\frac{d}{dt}e^x = e^x
\end{equation}
The exponential function can be written as a power series
\begin{equation}
	e^x = \sum_{k=0}^{\infty} \frac{x^k}{k!}
\end{equation}
\section*{Implementing the exponential function}
There are different ways of how to implement the exponential function, in c code. 
In math.h there is an exponential function, where we use exp(x), however there could be a reason for not using this implementation, and might want to make your own.
One implementation could be the so called quick and dirty (QAD) implementation described below.
\\
\textit{double ex(double x){\\
	if(x $<$ 0) return 1/ex(-x);\\
	if(x $>$ 1./8) return pow(ex(x/2),2);\\
	return $1+x*(1+x/2*(1+x/3*(1+x/4*(1+x/5*(1+x/6*(1+x/7*(1+x/8*(1+x/9*(1+x/10)))))))))$;
}
}
\\


\begin{figure}[h]
	% GNUPLOT: LaTeX picture
\setlength{\unitlength}{0.240900pt}
\ifx\plotpoint\undefined\newsavebox{\plotpoint}\fi
\sbox{\plotpoint}{\rule[-0.200pt]{0.400pt}{0.400pt}}%
\begin{picture}(1500,900)(0,0)
\sbox{\plotpoint}{\rule[-0.200pt]{0.400pt}{0.400pt}}%
\put(171.0,131.0){\rule[-0.200pt]{4.818pt}{0.400pt}}
\put(151,131){\makebox(0,0)[r]{$0$}}
\put(1419.0,131.0){\rule[-0.200pt]{4.818pt}{0.400pt}}
\put(171.0,252.0){\rule[-0.200pt]{4.818pt}{0.400pt}}
\put(151,252){\makebox(0,0)[r]{$500$}}
\put(1419.0,252.0){\rule[-0.200pt]{4.818pt}{0.400pt}}
\put(171.0,373.0){\rule[-0.200pt]{4.818pt}{0.400pt}}
\put(151,373){\makebox(0,0)[r]{$1000$}}
\put(1419.0,373.0){\rule[-0.200pt]{4.818pt}{0.400pt}}
\put(171.0,495.0){\rule[-0.200pt]{4.818pt}{0.400pt}}
\put(151,495){\makebox(0,0)[r]{$1500$}}
\put(1419.0,495.0){\rule[-0.200pt]{4.818pt}{0.400pt}}
\put(171.0,616.0){\rule[-0.200pt]{4.818pt}{0.400pt}}
\put(151,616){\makebox(0,0)[r]{$2000$}}
\put(1419.0,616.0){\rule[-0.200pt]{4.818pt}{0.400pt}}
\put(171.0,737.0){\rule[-0.200pt]{4.818pt}{0.400pt}}
\put(151,737){\makebox(0,0)[r]{$2500$}}
\put(1419.0,737.0){\rule[-0.200pt]{4.818pt}{0.400pt}}
\put(171.0,858.0){\rule[-0.200pt]{4.818pt}{0.400pt}}
\put(151,858){\makebox(0,0)[r]{$3000$}}
\put(1419.0,858.0){\rule[-0.200pt]{4.818pt}{0.400pt}}
\put(171.0,131.0){\rule[-0.200pt]{0.400pt}{4.818pt}}
\put(171,90){\makebox(0,0){$0$}}
\put(171.0,838.0){\rule[-0.200pt]{0.400pt}{4.818pt}}
\put(330.0,131.0){\rule[-0.200pt]{0.400pt}{4.818pt}}
\put(330,90){\makebox(0,0){$1$}}
\put(330.0,838.0){\rule[-0.200pt]{0.400pt}{4.818pt}}
\put(488.0,131.0){\rule[-0.200pt]{0.400pt}{4.818pt}}
\put(488,90){\makebox(0,0){$2$}}
\put(488.0,838.0){\rule[-0.200pt]{0.400pt}{4.818pt}}
\put(647.0,131.0){\rule[-0.200pt]{0.400pt}{4.818pt}}
\put(647,90){\makebox(0,0){$3$}}
\put(647.0,838.0){\rule[-0.200pt]{0.400pt}{4.818pt}}
\put(805.0,131.0){\rule[-0.200pt]{0.400pt}{4.818pt}}
\put(805,90){\makebox(0,0){$4$}}
\put(805.0,838.0){\rule[-0.200pt]{0.400pt}{4.818pt}}
\put(964.0,131.0){\rule[-0.200pt]{0.400pt}{4.818pt}}
\put(964,90){\makebox(0,0){$5$}}
\put(964.0,838.0){\rule[-0.200pt]{0.400pt}{4.818pt}}
\put(1122.0,131.0){\rule[-0.200pt]{0.400pt}{4.818pt}}
\put(1122,90){\makebox(0,0){$6$}}
\put(1122.0,838.0){\rule[-0.200pt]{0.400pt}{4.818pt}}
\put(1281.0,131.0){\rule[-0.200pt]{0.400pt}{4.818pt}}
\put(1281,90){\makebox(0,0){$7$}}
\put(1281.0,838.0){\rule[-0.200pt]{0.400pt}{4.818pt}}
\put(1439.0,131.0){\rule[-0.200pt]{0.400pt}{4.818pt}}
\put(1439,90){\makebox(0,0){$8$}}
\put(1439.0,838.0){\rule[-0.200pt]{0.400pt}{4.818pt}}
\put(171.0,131.0){\rule[-0.200pt]{0.400pt}{175.134pt}}
\put(171.0,131.0){\rule[-0.200pt]{305.461pt}{0.400pt}}
\put(1439.0,131.0){\rule[-0.200pt]{0.400pt}{175.134pt}}
\put(171.0,858.0){\rule[-0.200pt]{305.461pt}{0.400pt}}
\put(36,494){\rotatebox{-270}{\makebox(0,0){$y$}}
}\put(805,29){\makebox(0,0){$x$}}
\put(1279,817){\makebox(0,0)[r]{QAD exp}}
\put(1299.0,817.0){\rule[-0.200pt]{24.090pt}{0.400pt}}
\put(181,131){\usebox{\plotpoint}}
\put(280,130.67){\rule{2.409pt}{0.400pt}}
\multiput(280.00,130.17)(5.000,1.000){2}{\rule{1.204pt}{0.400pt}}
\put(181.0,131.0){\rule[-0.200pt]{23.849pt}{0.400pt}}
\put(458,131.67){\rule{2.409pt}{0.400pt}}
\multiput(458.00,131.17)(5.000,1.000){2}{\rule{1.204pt}{0.400pt}}
\put(290.0,132.0){\rule[-0.200pt]{40.471pt}{0.400pt}}
\put(538,132.67){\rule{2.168pt}{0.400pt}}
\multiput(538.00,132.17)(4.500,1.000){2}{\rule{1.084pt}{0.400pt}}
\put(468.0,133.0){\rule[-0.200pt]{16.863pt}{0.400pt}}
\put(587,133.67){\rule{2.409pt}{0.400pt}}
\multiput(587.00,133.17)(5.000,1.000){2}{\rule{1.204pt}{0.400pt}}
\put(547.0,134.0){\rule[-0.200pt]{9.636pt}{0.400pt}}
\put(627,134.67){\rule{2.409pt}{0.400pt}}
\multiput(627.00,134.17)(5.000,1.000){2}{\rule{1.204pt}{0.400pt}}
\put(597.0,135.0){\rule[-0.200pt]{7.227pt}{0.400pt}}
\put(656,135.67){\rule{2.409pt}{0.400pt}}
\multiput(656.00,135.17)(5.000,1.000){2}{\rule{1.204pt}{0.400pt}}
\put(637.0,136.0){\rule[-0.200pt]{4.577pt}{0.400pt}}
\put(686,136.67){\rule{2.409pt}{0.400pt}}
\multiput(686.00,136.17)(5.000,1.000){2}{\rule{1.204pt}{0.400pt}}
\put(666.0,137.0){\rule[-0.200pt]{4.818pt}{0.400pt}}
\put(706,137.67){\rule{2.409pt}{0.400pt}}
\multiput(706.00,137.17)(5.000,1.000){2}{\rule{1.204pt}{0.400pt}}
\put(696.0,138.0){\rule[-0.200pt]{2.409pt}{0.400pt}}
\put(726,138.67){\rule{2.409pt}{0.400pt}}
\multiput(726.00,138.17)(5.000,1.000){2}{\rule{1.204pt}{0.400pt}}
\put(716.0,139.0){\rule[-0.200pt]{2.409pt}{0.400pt}}
\put(746,139.67){\rule{2.168pt}{0.400pt}}
\multiput(746.00,139.17)(4.500,1.000){2}{\rule{1.084pt}{0.400pt}}
\put(736.0,140.0){\rule[-0.200pt]{2.409pt}{0.400pt}}
\put(765,140.67){\rule{2.409pt}{0.400pt}}
\multiput(765.00,140.17)(5.000,1.000){2}{\rule{1.204pt}{0.400pt}}
\put(775,141.67){\rule{2.409pt}{0.400pt}}
\multiput(775.00,141.17)(5.000,1.000){2}{\rule{1.204pt}{0.400pt}}
\put(755.0,141.0){\rule[-0.200pt]{2.409pt}{0.400pt}}
\put(795,142.67){\rule{2.409pt}{0.400pt}}
\multiput(795.00,142.17)(5.000,1.000){2}{\rule{1.204pt}{0.400pt}}
\put(805,143.67){\rule{2.409pt}{0.400pt}}
\multiput(805.00,143.17)(5.000,1.000){2}{\rule{1.204pt}{0.400pt}}
\put(815,144.67){\rule{2.409pt}{0.400pt}}
\multiput(815.00,144.17)(5.000,1.000){2}{\rule{1.204pt}{0.400pt}}
\put(825,145.67){\rule{2.409pt}{0.400pt}}
\multiput(825.00,145.17)(5.000,1.000){2}{\rule{1.204pt}{0.400pt}}
\put(835,146.67){\rule{2.409pt}{0.400pt}}
\multiput(835.00,146.17)(5.000,1.000){2}{\rule{1.204pt}{0.400pt}}
\put(845,147.67){\rule{2.409pt}{0.400pt}}
\multiput(845.00,147.17)(5.000,1.000){2}{\rule{1.204pt}{0.400pt}}
\put(855,148.67){\rule{2.168pt}{0.400pt}}
\multiput(855.00,148.17)(4.500,1.000){2}{\rule{1.084pt}{0.400pt}}
\put(864,149.67){\rule{2.409pt}{0.400pt}}
\multiput(864.00,149.17)(5.000,1.000){2}{\rule{1.204pt}{0.400pt}}
\put(874,151.17){\rule{2.100pt}{0.400pt}}
\multiput(874.00,150.17)(5.641,2.000){2}{\rule{1.050pt}{0.400pt}}
\put(884,152.67){\rule{2.409pt}{0.400pt}}
\multiput(884.00,152.17)(5.000,1.000){2}{\rule{1.204pt}{0.400pt}}
\put(894,154.17){\rule{2.100pt}{0.400pt}}
\multiput(894.00,153.17)(5.641,2.000){2}{\rule{1.050pt}{0.400pt}}
\put(904,155.67){\rule{2.409pt}{0.400pt}}
\multiput(904.00,155.17)(5.000,1.000){2}{\rule{1.204pt}{0.400pt}}
\put(914,157.17){\rule{2.100pt}{0.400pt}}
\multiput(914.00,156.17)(5.641,2.000){2}{\rule{1.050pt}{0.400pt}}
\put(924,159.17){\rule{2.100pt}{0.400pt}}
\multiput(924.00,158.17)(5.641,2.000){2}{\rule{1.050pt}{0.400pt}}
\put(934,161.17){\rule{2.100pt}{0.400pt}}
\multiput(934.00,160.17)(5.641,2.000){2}{\rule{1.050pt}{0.400pt}}
\put(944,163.17){\rule{2.100pt}{0.400pt}}
\multiput(944.00,162.17)(5.641,2.000){2}{\rule{1.050pt}{0.400pt}}
\put(954,165.17){\rule{2.100pt}{0.400pt}}
\multiput(954.00,164.17)(5.641,2.000){2}{\rule{1.050pt}{0.400pt}}
\put(964,167.17){\rule{1.900pt}{0.400pt}}
\multiput(964.00,166.17)(5.056,2.000){2}{\rule{0.950pt}{0.400pt}}
\multiput(973.00,169.61)(2.025,0.447){3}{\rule{1.433pt}{0.108pt}}
\multiput(973.00,168.17)(7.025,3.000){2}{\rule{0.717pt}{0.400pt}}
\put(983,172.17){\rule{2.100pt}{0.400pt}}
\multiput(983.00,171.17)(5.641,2.000){2}{\rule{1.050pt}{0.400pt}}
\multiput(993.00,174.61)(2.025,0.447){3}{\rule{1.433pt}{0.108pt}}
\multiput(993.00,173.17)(7.025,3.000){2}{\rule{0.717pt}{0.400pt}}
\multiput(1003.00,177.61)(2.025,0.447){3}{\rule{1.433pt}{0.108pt}}
\multiput(1003.00,176.17)(7.025,3.000){2}{\rule{0.717pt}{0.400pt}}
\multiput(1013.00,180.61)(2.025,0.447){3}{\rule{1.433pt}{0.108pt}}
\multiput(1013.00,179.17)(7.025,3.000){2}{\rule{0.717pt}{0.400pt}}
\multiput(1023.00,183.60)(1.358,0.468){5}{\rule{1.100pt}{0.113pt}}
\multiput(1023.00,182.17)(7.717,4.000){2}{\rule{0.550pt}{0.400pt}}
\multiput(1033.00,187.61)(2.025,0.447){3}{\rule{1.433pt}{0.108pt}}
\multiput(1033.00,186.17)(7.025,3.000){2}{\rule{0.717pt}{0.400pt}}
\multiput(1043.00,190.60)(1.358,0.468){5}{\rule{1.100pt}{0.113pt}}
\multiput(1043.00,189.17)(7.717,4.000){2}{\rule{0.550pt}{0.400pt}}
\multiput(1053.00,194.60)(1.358,0.468){5}{\rule{1.100pt}{0.113pt}}
\multiput(1053.00,193.17)(7.717,4.000){2}{\rule{0.550pt}{0.400pt}}
\multiput(1063.00,198.59)(0.933,0.477){7}{\rule{0.820pt}{0.115pt}}
\multiput(1063.00,197.17)(7.298,5.000){2}{\rule{0.410pt}{0.400pt}}
\multiput(1072.00,203.60)(1.358,0.468){5}{\rule{1.100pt}{0.113pt}}
\multiput(1072.00,202.17)(7.717,4.000){2}{\rule{0.550pt}{0.400pt}}
\multiput(1082.00,207.59)(1.044,0.477){7}{\rule{0.900pt}{0.115pt}}
\multiput(1082.00,206.17)(8.132,5.000){2}{\rule{0.450pt}{0.400pt}}
\multiput(1092.00,212.59)(1.044,0.477){7}{\rule{0.900pt}{0.115pt}}
\multiput(1092.00,211.17)(8.132,5.000){2}{\rule{0.450pt}{0.400pt}}
\multiput(1102.00,217.59)(0.852,0.482){9}{\rule{0.767pt}{0.116pt}}
\multiput(1102.00,216.17)(8.409,6.000){2}{\rule{0.383pt}{0.400pt}}
\multiput(1112.00,223.59)(0.852,0.482){9}{\rule{0.767pt}{0.116pt}}
\multiput(1112.00,222.17)(8.409,6.000){2}{\rule{0.383pt}{0.400pt}}
\multiput(1122.00,229.59)(0.852,0.482){9}{\rule{0.767pt}{0.116pt}}
\multiput(1122.00,228.17)(8.409,6.000){2}{\rule{0.383pt}{0.400pt}}
\multiput(1132.00,235.59)(0.721,0.485){11}{\rule{0.671pt}{0.117pt}}
\multiput(1132.00,234.17)(8.606,7.000){2}{\rule{0.336pt}{0.400pt}}
\multiput(1142.00,242.59)(0.721,0.485){11}{\rule{0.671pt}{0.117pt}}
\multiput(1142.00,241.17)(8.606,7.000){2}{\rule{0.336pt}{0.400pt}}
\multiput(1152.00,249.59)(0.626,0.488){13}{\rule{0.600pt}{0.117pt}}
\multiput(1152.00,248.17)(8.755,8.000){2}{\rule{0.300pt}{0.400pt}}
\multiput(1162.00,257.59)(0.626,0.488){13}{\rule{0.600pt}{0.117pt}}
\multiput(1162.00,256.17)(8.755,8.000){2}{\rule{0.300pt}{0.400pt}}
\multiput(1172.00,265.59)(0.560,0.488){13}{\rule{0.550pt}{0.117pt}}
\multiput(1172.00,264.17)(7.858,8.000){2}{\rule{0.275pt}{0.400pt}}
\multiput(1181.00,273.59)(0.553,0.489){15}{\rule{0.544pt}{0.118pt}}
\multiput(1181.00,272.17)(8.870,9.000){2}{\rule{0.272pt}{0.400pt}}
\multiput(1191.00,282.58)(0.495,0.491){17}{\rule{0.500pt}{0.118pt}}
\multiput(1191.00,281.17)(8.962,10.000){2}{\rule{0.250pt}{0.400pt}}
\multiput(1201.58,292.00)(0.491,0.547){17}{\rule{0.118pt}{0.540pt}}
\multiput(1200.17,292.00)(10.000,9.879){2}{\rule{0.400pt}{0.270pt}}
\multiput(1211.58,303.00)(0.491,0.547){17}{\rule{0.118pt}{0.540pt}}
\multiput(1210.17,303.00)(10.000,9.879){2}{\rule{0.400pt}{0.270pt}}
\multiput(1221.58,314.00)(0.491,0.547){17}{\rule{0.118pt}{0.540pt}}
\multiput(1220.17,314.00)(10.000,9.879){2}{\rule{0.400pt}{0.270pt}}
\multiput(1231.58,325.00)(0.491,0.652){17}{\rule{0.118pt}{0.620pt}}
\multiput(1230.17,325.00)(10.000,11.713){2}{\rule{0.400pt}{0.310pt}}
\multiput(1241.58,338.00)(0.491,0.652){17}{\rule{0.118pt}{0.620pt}}
\multiput(1240.17,338.00)(10.000,11.713){2}{\rule{0.400pt}{0.310pt}}
\multiput(1251.58,351.00)(0.491,0.756){17}{\rule{0.118pt}{0.700pt}}
\multiput(1250.17,351.00)(10.000,13.547){2}{\rule{0.400pt}{0.350pt}}
\multiput(1261.58,366.00)(0.491,0.756){17}{\rule{0.118pt}{0.700pt}}
\multiput(1260.17,366.00)(10.000,13.547){2}{\rule{0.400pt}{0.350pt}}
\multiput(1271.58,381.00)(0.491,0.808){17}{\rule{0.118pt}{0.740pt}}
\multiput(1270.17,381.00)(10.000,14.464){2}{\rule{0.400pt}{0.370pt}}
\multiput(1281.59,397.00)(0.489,0.961){15}{\rule{0.118pt}{0.856pt}}
\multiput(1280.17,397.00)(9.000,15.224){2}{\rule{0.400pt}{0.428pt}}
\multiput(1290.58,414.00)(0.491,0.912){17}{\rule{0.118pt}{0.820pt}}
\multiput(1289.17,414.00)(10.000,16.298){2}{\rule{0.400pt}{0.410pt}}
\multiput(1300.58,432.00)(0.491,1.017){17}{\rule{0.118pt}{0.900pt}}
\multiput(1299.17,432.00)(10.000,18.132){2}{\rule{0.400pt}{0.450pt}}
\multiput(1310.58,452.00)(0.491,1.017){17}{\rule{0.118pt}{0.900pt}}
\multiput(1309.17,452.00)(10.000,18.132){2}{\rule{0.400pt}{0.450pt}}
\multiput(1320.58,472.00)(0.491,1.121){17}{\rule{0.118pt}{0.980pt}}
\multiput(1319.17,472.00)(10.000,19.966){2}{\rule{0.400pt}{0.490pt}}
\multiput(1330.58,494.00)(0.491,1.225){17}{\rule{0.118pt}{1.060pt}}
\multiput(1329.17,494.00)(10.000,21.800){2}{\rule{0.400pt}{0.530pt}}
\multiput(1340.58,518.00)(0.491,1.277){17}{\rule{0.118pt}{1.100pt}}
\multiput(1339.17,518.00)(10.000,22.717){2}{\rule{0.400pt}{0.550pt}}
\multiput(1350.58,543.00)(0.491,1.329){17}{\rule{0.118pt}{1.140pt}}
\multiput(1349.17,543.00)(10.000,23.634){2}{\rule{0.400pt}{0.570pt}}
\multiput(1360.58,569.00)(0.491,1.433){17}{\rule{0.118pt}{1.220pt}}
\multiput(1359.17,569.00)(10.000,25.468){2}{\rule{0.400pt}{0.610pt}}
\multiput(1370.58,597.00)(0.491,1.538){17}{\rule{0.118pt}{1.300pt}}
\multiput(1369.17,597.00)(10.000,27.302){2}{\rule{0.400pt}{0.650pt}}
\multiput(1380.59,627.00)(0.489,1.893){15}{\rule{0.118pt}{1.567pt}}
\multiput(1379.17,627.00)(9.000,29.748){2}{\rule{0.400pt}{0.783pt}}
\multiput(1389.58,660.00)(0.491,1.746){17}{\rule{0.118pt}{1.460pt}}
\multiput(1388.17,660.00)(10.000,30.970){2}{\rule{0.400pt}{0.730pt}}
\multiput(1399.58,694.00)(0.491,1.850){17}{\rule{0.118pt}{1.540pt}}
\multiput(1398.17,694.00)(10.000,32.804){2}{\rule{0.400pt}{0.770pt}}
\multiput(1409.58,730.00)(0.491,2.007){17}{\rule{0.118pt}{1.660pt}}
\multiput(1408.17,730.00)(10.000,35.555){2}{\rule{0.400pt}{0.830pt}}
\multiput(1419.58,769.00)(0.491,2.111){17}{\rule{0.118pt}{1.740pt}}
\multiput(1418.17,769.00)(10.000,37.389){2}{\rule{0.400pt}{0.870pt}}
\multiput(1429.58,810.00)(0.491,2.215){17}{\rule{0.118pt}{1.820pt}}
\multiput(1428.17,810.00)(10.000,39.222){2}{\rule{0.400pt}{0.910pt}}
\put(785.0,143.0){\rule[-0.200pt]{2.409pt}{0.400pt}}
\put(1279,776){\makebox(0,0)[r]{exp}}
\multiput(1299,776)(20.756,0.000){5}{\usebox{\plotpoint}}
\put(1399,776){\usebox{\plotpoint}}
\put(181,131){\usebox{\plotpoint}}
\put(181.00,131.00){\usebox{\plotpoint}}
\put(201.76,131.00){\usebox{\plotpoint}}
\put(222.51,131.00){\usebox{\plotpoint}}
\put(243.27,131.00){\usebox{\plotpoint}}
\put(264.02,131.00){\usebox{\plotpoint}}
\put(284.75,131.48){\usebox{\plotpoint}}
\put(305.48,132.00){\usebox{\plotpoint}}
\put(326.24,132.00){\usebox{\plotpoint}}
\put(346.99,132.00){\usebox{\plotpoint}}
\put(367.75,132.00){\usebox{\plotpoint}}
\put(388.51,132.00){\usebox{\plotpoint}}
\put(409.26,132.00){\usebox{\plotpoint}}
\put(430.02,132.00){\usebox{\plotpoint}}
\put(450.77,132.00){\usebox{\plotpoint}}
\put(471.48,133.00){\usebox{\plotpoint}}
\put(492.23,133.00){\usebox{\plotpoint}}
\put(512.99,133.00){\usebox{\plotpoint}}
\put(533.74,133.00){\usebox{\plotpoint}}
\put(554.44,134.00){\usebox{\plotpoint}}
\put(575.20,134.00){\usebox{\plotpoint}}
\put(595.91,134.89){\usebox{\plotpoint}}
\put(616.66,135.00){\usebox{\plotpoint}}
\put(637.37,136.00){\usebox{\plotpoint}}
\put(658.11,136.21){\usebox{\plotpoint}}
\put(678.83,137.00){\usebox{\plotpoint}}
\put(699.53,138.00){\usebox{\plotpoint}}
\put(720.24,139.00){\usebox{\plotpoint}}
\put(740.94,140.00){\usebox{\plotpoint}}
\put(761.64,141.00){\usebox{\plotpoint}}
\put(782.31,142.73){\usebox{\plotpoint}}
\put(803.02,143.80){\usebox{\plotpoint}}
\put(823.67,145.87){\usebox{\plotpoint}}
\put(844.32,147.93){\usebox{\plotpoint}}
\put(864.96,150.10){\usebox{\plotpoint}}
\put(885.47,153.15){\usebox{\plotpoint}}
\put(905.97,156.20){\usebox{\plotpoint}}
\put(926.44,159.49){\usebox{\plotpoint}}
\put(946.79,163.56){\usebox{\plotpoint}}
\put(967.13,167.70){\usebox{\plotpoint}}
\put(987.22,172.84){\usebox{\plotpoint}}
\put(1007.24,178.27){\usebox{\plotpoint}}
\put(1026.99,184.60){\usebox{\plotpoint}}
\put(1046.57,191.43){\usebox{\plotpoint}}
\put(1065.67,199.48){\usebox{\plotpoint}}
\put(1084.46,208.23){\usebox{\plotpoint}}
\put(1102.98,217.59){\usebox{\plotpoint}}
\put(1120.78,228.27){\usebox{\plotpoint}}
\put(1138.28,239.40){\usebox{\plotpoint}}
\put(1155.13,251.50){\usebox{\plotpoint}}
\put(1171.34,264.47){\usebox{\plotpoint}}
\put(1186.85,278.26){\usebox{\plotpoint}}
\put(1201.69,292.76){\usebox{\plotpoint}}
\put(1215.65,308.12){\usebox{\plotpoint}}
\put(1229.61,323.47){\usebox{\plotpoint}}
\put(1242.40,339.82){\usebox{\plotpoint}}
\put(1254.69,356.53){\usebox{\plotpoint}}
\put(1266.20,373.80){\usebox{\plotpoint}}
\put(1277.41,391.26){\usebox{\plotpoint}}
\put(1287.55,409.36){\usebox{\plotpoint}}
\put(1297.53,427.56){\usebox{\plotpoint}}
\put(1307.01,446.02){\usebox{\plotpoint}}
\put(1316.29,464.58){\usebox{\plotpoint}}
\put(1325.16,483.35){\usebox{\plotpoint}}
\put(1333.48,502.36){\usebox{\plotpoint}}
\multiput(1340,518)(7.708,19.271){2}{\usebox{\plotpoint}}
\put(1356.60,560.17){\usebox{\plotpoint}}
\put(1363.80,579.63){\usebox{\plotpoint}}
\multiput(1370,597)(6.563,19.690){2}{\usebox{\plotpoint}}
\multiput(1380,627)(5.461,20.024){2}{\usebox{\plotpoint}}
\put(1394.50,678.72){\usebox{\plotpoint}}
\multiput(1399,694)(5.555,19.998){2}{\usebox{\plotpoint}}
\multiput(1409,730)(5.155,20.105){2}{\usebox{\plotpoint}}
\multiput(1419,769)(4.918,20.164){2}{\usebox{\plotpoint}}
\multiput(1429,810)(4.701,20.216){2}{\usebox{\plotpoint}}
\put(1439,853){\usebox{\plotpoint}}
\put(171.0,131.0){\rule[-0.200pt]{0.400pt}{175.134pt}}
\put(171.0,131.0){\rule[-0.200pt]{305.461pt}{0.400pt}}
\put(1439.0,131.0){\rule[-0.200pt]{0.400pt}{175.134pt}}
\put(171.0,858.0){\rule[-0.200pt]{305.461pt}{0.400pt}}
\end{picture}

	\caption{exponential function plot}
	\label{fig:gpl}
\end{figure}

\end{document}
